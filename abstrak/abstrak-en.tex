\begin{center}
  \large\textbf{ABSTRACT}
\end{center}

\addcontentsline{toc}{chapter}{ABSTRACT}

\vspace{2ex}

\begingroup
% Menghilangkan padding
\setlength{\tabcolsep}{0pt}

\noindent
\begin{tabularx}{\textwidth}{l >{\centering}m{3em} X}
  \emph{Name}     & : & {Abel Marcel Renwarin}         \\

  \emph{Title}    & : & {CAPITAL MARKET PREDICTION USING \textit{TIME SERIES TRANSFORMER}}   \\

  \emph{Advisors} & : & 1. {Reza Fuad Rachmadi, S.T., M.T., Ph.D}   \\
                  &   & 2. {Dr. Surya Sumpeno, S.T., M.Sc.} \\
\end{tabularx}
\endgroup

% Ubah paragraf berikut dengan abstrak dari tugas akhir dalam Bahasa Inggris
The movement of capital market prices is a dynamic phenomenon influenced by various economic, political, and social factors. Accurate prediction of capital market prices is important in making better investment decisions. This study aims to develop a capital market price prediction model using Time Series Transformer (TST). TST is used to improve the model's ability to capture more complex dynamic patterns by considering temporal shifts. This study uses historical forex price data available in open sources as the object of study, and will evaluate the TST model in predicting future prices. Model evaluation is carried out using prediction accuracy metrics and Mean Absolute Error (MAE). It is expected that the results of this study can contribute to the development of more accurate and efficient capital market prediction methods.

% Ubah kata-kata berikut dengan kata kunci dari tugas akhir dalam Bahasa Inggris
\emph{Keywords: Stock Market Price Prediction, Forex, Time Series Transformer(TST)}.
