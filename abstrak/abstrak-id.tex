\begin{center}
  \large\textbf{ABSTRAK}
\end{center}

\addcontentsline{toc}{chapter}{ABSTRAK}

\vspace{2ex}

\begingroup
% Menghilangkan padding
\setlength{\tabcolsep}{0pt}

\noindent
\begin{tabularx}{\textwidth}{l >{\centering}m{2em} X}
  Nama Mahasiswa    & : & {Abel Marcel Renwarin}         \\

  Judul Tugas Akhir & : & {PREDIKSI HARGA PASAR MODAL MENGGUNAKAN \textit{TIME SERIES TRANSFORMER}}      \\

  Pembimbing        & : & 1. {Reza Fuad Rachmadi, S.T., M.T., Ph.D}   \\
                    &   & 2. {Dr. Surya Sumpeno, S.T., M.Sc.} \\
\end{tabularx}
\endgroup

% Ubah paragraf berikut dengan abstrak dari tugas akhir
Pergerakan harga pasar modal merupakan fenomena dinamis yang dipengaruhi oleh berbagai faktor ekonomi, politik, dan sosial. Prediksi yang akurat terhadap harga pasar modal menjadi penting dalam pengambilan keputusan investasi yang lebih baik. Penelitian ini bertujuan untuk mengembangkan model prediksi harga pasar modal menggunakan Time Series Transformer (TST). TST digunakan untuk meningkatkan kemampuan model dalam menangkap pola dinamis yang lebih kompleks dengan mempertimbangkan pergeseran temporal. Penelitian ini menggunakan data historis harga forex yang tersedia di sumber terbuka sebagai objek studi, dan akan mengevaluasi model TST dalam memprediksi harga di masa depan. Evaluasi model dilakukan menggunakan metrik akurasi prediksi dan Mean Absolute Error (MAE). Diharapkan hasil dari penelitian ini dapat memberikan kontribusi pada pengembangan metode prediksi pasar modal yang lebih akurat dan efisien.

% Ubah kata-kata berikut dengan kata kunci dari tugas akhir
Kata Kunci: \emph{Prediksi Pasar Modal, Forex, TST(Time Series Transformer)}
